\documentclass[dvipdfmx]{beamer}

%Beamerの設定
% \usetheme{Boadilla}
\usetheme{Antibes}

\usepackage{comment}

%Beamerフォント設定
\usepackage{newtxtext,newtxmath} % TXフォント
\usepackage{sfmath}
\DeclareSymbolFont{operators}{OT1}{\sfdefault}{m}{n}
\SetSymbolFont{operators}{bold}{OT1}{\sfdefault}{b}{n}
\usefonttheme{structurebold} % タイトル部を太字
\setbeamerfont{alerted text}{series=\bfseries} % Alertを太字
\setbeamerfont{section in toc}{series=\mdseries} % 目次は太字にしない
\setbeamerfont{frametitle}{size=\Large} % フレームタイトル文字サイズ
\setbeamerfont{title}{size=\LARGE} % タイトル文字サイズ
\setbeamerfont{date}{size=\small}  % 日付文字サイズ

% Babel (日本語の場合のみ・英語の場合は不要)
\uselanguage{japanese}
\languagepath{japanese}
\deftranslation[to=japanese]{Theorem}{定理}
\deftranslation[to=japanese]{Lemma}{補題}
\deftranslation[to=japanese]{Example}{例}
\deftranslation[to=japanese]{Examples}{例}
\deftranslation[to=japanese]{Definition}{定義}
\deftranslation[to=japanese]{Definitions}{定義}
\deftranslation[to=japanese]{Problem}{問題}
\deftranslation[to=japanese]{Solution}{解}
\deftranslation[to=japanese]{Fact}{事実}
\deftranslation[to=japanese]{Proof}{証明}
\def\proofname{証明}

%Beamer色設定
\definecolor{UniBlue}{RGB}{0,150,200}
\definecolor{AlertOrange}{RGB}{255,76,0}
\definecolor{AlmostBlack}{RGB}{38,38,38}
\setbeamercolor{normal text}{fg=AlmostBlack}  % 本文カラー
\setbeamercolor{structure}{fg=UniBlue} % 見出しカラー
% \setbeamercolor{block title}{fg=UniBlue!50!black} % ブロック部分タイトルカラー
\setbeamercolor{alerted text}{fg=AlertOrange} % \alert 文字カラー
\mode<beamer>{
    \definecolor{BackGroundGray}{RGB}{254,254,254}
    \setbeamercolor{background canvas}{bg=BackGroundGray} % スライドモードのみ背景をわずかにグレーにする
}

%フラットデザイン化
\setbeamertemplate{blocks}[rounded] % Blockの影を消す
\useinnertheme{circles} % 箇条書きをシンプルに
\setbeamertemplate{navigation symbols}{} % ナビゲーションシンボルを消す
\setbeamertemplate{footline}[frame number] % フッターはスライド番号のみ

% section毎に目次の表示
\AtBeginSection[]{
    \begin{frame}
        \tableofcontents[sectionstyle=show/shaded,subsectionstyle=show/shaded/hide]
    \end{frame}
}

\title{量子モンテカルロ法}
\author{齊藤}

\begin{document}

\maketitle
\frame{\tableofcontents[hideallsubsections]}

\section{目的}
\begin{frame}{目的}
    シュレディンガー方程式を解いて、調和振動子ポテンシャルや水素原子などの基底状態・励起状態のエネルギーを求めたい。

\end{frame}

    \section{シュレディンガー方程式と拡散方程式}

    \begin{frame}
      \begin{block}{シュレディンガー方程式}

        \begin{equation}
            i\hbar\dfrac{\partial \Psi(x,t)}{\partial t} = -\dfrac{\hbar^2}{2m}\dfrac{\partial^2 \Psi(x,t)}{\partial x^2} + V(x)\Psi(x,t)
        \end{equation}

      \end{block}

        $\tau = \dfrac{it}{\hbar}$と変数変換すると、

        \begin{block}{虚時間のシュレディンガー方程式}
            \begin{equation}
                \label{tmp1}
        \dfrac{\partial \Psi(x,\tau)}{\partial \tau} = \dfrac{\hbar^2}{2m}\dfrac{\partial^2 \Psi(x,\tau)}{\partial x^2} - V(x)\Psi(x,\tau)
            \end{equation}
        \end{block}
        これは、
        \begin{block}{拡散方程式}
            \begin{equation}
                \dfrac{\partial f(x,t)}{\partial t} = D \dfrac{\partial^2 f(x,t)}{\partial x^2}
                ,~~~~D = \frac{\Delta x^2}{2\Delta t}
            \end{equation}
        \end{block}
        似ていることがわかる。

        \alert{拡散方程式を解く手法が使えそう。}
    \end{frame}

    \begin{frame}
      拡散方程式を解く手法の一つであるランダムウォークを用いる。
      \begin{block}{ランダムウォーク}
          粒子を空間上に配置し、粒子をランダムに動かすことで系の状態を発展させ、定常状態にする。

          $\to$ 定常状態のエネルギーを求める。
      \end{block}

      \begin{exampleblock}{例:室温}
        空気中の分子が温度によってランダムに動くことによって、熱が拡散していく。

        $\to$ 十分に拡散すると最終的に室温は一定になる。(定常状態)
      \end{exampleblock}

    \end{frame}

    \begin{frame}
      式を観察すると、
        \begin{block}{虚時間のシュレディンガー方程式}
            \begin{equation}
                \label{tmp1}
        \dfrac{\partial \Psi(x,\tau)}{\partial \tau} = \dfrac{\hbar^2}{2m}\dfrac{\partial^2 \Psi(x,\tau)}{\partial x^2} - V(x)\Psi(x,\tau) \nonumber
            \end{equation}
        \end{block}

        \begin{itemize}
            \item 右辺第$1$項は粒子の拡散 (拡散過程)
            \item 右辺第$2$項はポテンシャル$V$の符号によって粒子が増減(分岐過程)
        \end{itemize}

        を表している。

    \end{frame}

    \section{ランダムウォーク量子モンテカルロ法}

    \subsection{準備}
    \begin{frame}{\insertsubsection}
      シュレディンガー方程式の一般解は虚時間に対して、
      \begin{equation}
        \Psi(x,\tau) = \sum_{n=0}^\infty c_n \phi_n(x) e^{-E_n\tau}
      \end{equation}
      ここで、エネルギーは
      \begin{equation}
        E_0 \leq E_1 \leq E_2 \leq \cdots
      \end{equation}
      なので、$\tau \to \infty$では$c_0$の項が支配的になる。
      \begin{equation}
        \label{tmp2}
        \Psi(x,\tau\to\infty) \approx c_0 \phi_0(x) e^{-E_0\tau}
      \end{equation}
      しかし、$E_0 \neq 0$なので、$\tau\to\infty$で$\Psi$が$0$もしくは$\infty$になってしまう。
      $\to$ とてもまずい。
    \end{frame}


    \begin{frame}
        $E_0 - V_{ref} \approx 0$になるような$V_{ref}$を用意する。

        すると、$V(x) \to V(x)-V_{ref}$とシフトするので

        (\ref{tmp1})は
        \begin{equation}
          \label{tmp3}
          \dfrac{\partial \Psi(x,\tau)}{\partial \tau} = \dfrac{\hbar^2}{2m}\dfrac{\partial^2 \Psi(x,\tau)}{\partial x^2} - [V(x)-V_{ref}]\Psi(x,\tau)
        \end{equation}

        (\ref{tmp2})は
        \begin{equation}
          \label{tmp4}
          \Psi(x,\tau\to\infty) \approx c_0 \phi_0(x) e^{-(E_0-V_{ref})\tau}
        \end{equation}

        となる。

    \end{frame}

    \begin{frame}
      (\ref{tmp3})の両辺を$x$で積分する。$\frac{\partial\Psi}{\partial x}$は$|x|\to\infty$で消えるので
      $\int \frac{\partial^2\Psi}{\partial x^2} dx = 0$より、
      \begin{equation}
        \label{tmp5}
        \int \frac{\partial\Psi(x,\tau)}{\partial \tau} dx = -\int V(x)\Psi(x,\tau)dx + V_{ref}\int \Psi(x,\tau)dx
      \end{equation}
      (\ref{tmp4})を$\tau$で微分すると、
      \begin{equation}
        \label{tmp6}
        \dfrac{\partial \Psi(x,\tau)}{\partial \tau} = -(E_0 - V_{ref})\Psi(x,\tau)
      \end{equation}
      (\ref{tmp6})を(\ref{tmp5})に代入すると、
      \begin{equation}
        E_0 = \dfrac{\int V(x) \Psi(x,\tau) dx}{\int \Psi(x,\tau) dx}
      \end{equation}
      を得る。
    \end{frame}
    \begin{frame}
      ここで、$\Psi(x,\tau)dx$と点$x$での粒子の密度を関係付ける。

      時刻$\tau$で位置$x_i$に存在する粒子の数を$n_i$とすれば
      \begin{equation}
        E_0 = \langle V \rangle =  \dfrac{\sum_{i} n_i V(x_i)}{\sum_{i} n_i}
      \end{equation}
      つまり、全粒子数を$N$とすると、
      \begin{equation}
          E_0 = \langle V \rangle = \frac{1}{N}\sum_{i} n_i V(x_i)
      \end{equation}
      となる。

    \end{frame}


    \subsection{アルゴリズム}

    \begin{frame}{\insertsubsection}
        \begin{enumerate}
            \item 初期化
            \begin{itemize}
                \item $N_0$個の粒子を初期値$x_i$にランダムに置く。
                \item $V_{ave} = V_{ref} = \frac{1}{N_0} \sum_{i=1}^{N_0}V(x_i)$ を計算。
            \end{itemize}
            \item 全ての粒子について
            \begin{itemize}
                \item 参照エネルギー
                $V_{ref} = \frac{1}{N_0}\sum_{i=1}^{N_0}V(x[i])$
                を計算。
                \item 粒子をランダムに$\Delta s$だけ移動(拡散)させる。

                $x[i] = x[i] \pm \Delta s$

                \item $\Delta V = V(x_i) - V_{ref}$を計算。
                $r$を単位区間$[0,1]$の乱数とする。

                IF $\Delta V > 0$ and $r < \Delta V \Delta \tau$

                ~~~~ その粒子を取り除く。

                ELSE IF $\Delta V < 0$ and $r < -\Delta V \Delta \tau$

                ~~~~ 粒子を$1$個同じ位置に追加する。

                END IF
            \end{itemize}

        \end{enumerate}
    \end{frame}

    \begin{frame}
        \begin{enumerate}
            \setcounter{enumi}{2}
            \item $V_{ave}$と$V_{ref}$の更新

            $V_{ave} = \frac{1}{N}\sum_{i} n_i V(x_i)$

            $V_{ref} = V_{ave} - \frac{a}{N_0\Delta\tau}(N-N_0)$

            \item 上記$2$と$3$が$V_{ave}$がある値で不規則なゆらぎを示すまで繰り返す。

            \item $V_{ave}$の平均を取ることで基底状態のエネルギーを求めることができる。
        \end{enumerate}
        ここで、$\hbar = m = 1$という単位系を採用し、
        $(\Delta s)^2 = 2D\Delta \tau~(\tau \ll 1)$

        また、$N$は上記$2$によって増減した後の粒子数であり、$a$は粒子数$N$が一定になるように調整するためのパラメーターである。
    \end{frame}




    \section{拡散量子モンテカルロ法1}
    \subsection{準備}
    \begin{frame}{\insertsubsection}
      ハミルトニアン$\hat{H}$を$V_{ref}$だけシフトさせておくと
      \begin{equation}
        \hat{H} - V_{ref} = -\dfrac{\hbar^2}{2m}\dfrac{\partial^2}{\partial x^2} + V(x) - V_{ref}
      \end{equation}
      \begin{block}{虚時間のシュレディンガー方程式}
          \begin{equation}
              \label{tmp7}
      \dfrac{\partial \Psi(x,\tau)}{\partial \tau} = -(\hat{H}-V_{ref})\Psi(x,\tau) \nonumber
          \end{equation}
      \end{block}

      これの解は形式的に
      \begin{equation}
        \Psi(x,\tau) = e^{-(\hat{H}-V_{ref})\tau}\Psi(x,0)
      \end{equation}
      となる。

      つまり、$e^{-(\hat{H}-V_{ref})\tau}$が波動関数に作用することで時間を$\tau$だけ進めることがわかる。

    \end{frame}

    \begin{frame}
        この$e^{-(\hat{H}-V_{ref})\tau}$をGreen関数(伝達関数)と呼び、$G(\tau) = e^{-(\hat{H}-V_{ref})\tau}$とかく。

        演算子の指数関数は、行列の指数関数を意味する。

        というのも、量子力学において演算子は無限次元のベクトル(波動関数)に作用する無限次元の行列と考えられるため。
        \begin{block}{行列の指数関数}
            \begin{equation}
                e^A = \sum_{k=0}^\infty \dfrac{1}{k!}A^k = E + A + \frac{A^2}{2!} + \frac{A^3}{3!} + \cdots
            \end{equation}
        \end{block}


      \end{frame}

      \begin{frame}

        また、$G$は
        \begin{equation}
          \Psi(x',\tau) = \int G(x',x,\tau)\Psi(x,0) dx
        \end{equation}
        とも定義される。
        \begin{equation}
            G(x',x,\tau) = G(x,x',\tau)
        \end{equation}
        を満たす。(位置について対称)
      \end{frame}

    \begin{frame}
        $\hat{H}$の中にある運動エネルギー演算子$\hat{T}(=\frac{\hat{p}^2}{2m},\hat{p} = -i\hbar\frac{\partial}{\partial x})$
        とポテンシャルエネルギー演算子$\hat{V} = \frac{V(x') + V(x)}{2}$が互いに非可換であるため、演算子の指数関数の計算が困難。

        そのため
        \begin{block}{鈴木・トロッターの分解公式}
            \begin{align}
                G_{diff} &= e^{-\hat{T}\Delta\tau} \\
                G_{branch/2} &= e^{-\frac{1}{2}(\hat{V} - V_{ref})\Delta\tau} \\
                G(\tau) &= G_{branch/2}G_{diff}G_{branch/2}
            \end{align}
        \end{block}
        を用いる。

        $G_{diff}$が拡散過程、$G_{branch/2}$が分岐過程を表す。

        つまり、$G$は分岐$\to$拡散$\to$分岐の流れを表している。
    \end{frame}

    \begin{frame}
        $G_{diff}$は
        \begin{equation}
          \label{Gdiff_eq}
            \dfrac{\partial G_{diff}}{\partial \tau} = - \hat{T}G_{diff} = \dfrac{\hbar^2}{2m}\dfrac{\partial^2 G_{diff}}{\partial x^2}
        \end{equation}
        $G_{branch/2}$は
        \begin{equation}
          \label{Gbranch_eq}
            \dfrac{\partial G_{branch/2}}{\partial \tau} = -\dfrac{1}{2}(\hat{V} - V_{ref})G_{branch/2}
        \end{equation}
        の微分方程式を満たす。

        これらの解は

        \begin{align}
            G_{diff}(x',x,\Delta\tau) &= \dfrac{1}{\sqrt{4\pi D\Delta\tau}}
            e^{-\frac{(x'-x)^2}{4D\Delta\tau}} \\
            \label{Gbranch}
            G_{branch/2}(x',x,\Delta\tau) &= e^{-(\frac{1}{2}[V(x)+V(x')]-V_{ref})\Delta\tau}
        \end{align}
        ただし、$D=\dfrac{\hbar^2}{2m}$である。
    \end{frame}


    \subsection{アルゴリズム}

    \begin{frame}{\insertsubsection}
        \begin{enumerate}
            \item 初期化
            \begin{itemize}
                \item $N_0$個の粒子を初期値$x_i$にランダムに置く。
                \item $V_{ave} = V_{ref} = \frac{1}{N_0} \sum_{i=1}^{N_0}V(x_i)$ を計算。
            \end{itemize}

            \item 全ての粒子について
            \begin{itemize}
                \item 分散$2D\tau$、平均$0$のガウス分布から乱数$r$を生成する。
                \item $x_i = x_i + r$と位置を更新する。($G_{diff}$による拡散過程に対応)
                \item $w(x\to x',\tau) = G_{branch/2}(x',x,\Delta\tau)$を計算する。
                \item 単位区間$[0,1]$から乱数$r$を生成する。$w+r$の整数部分を$w'$とする。

                IF $w' > 1$

                ~~~~粒子を$w'-1$個同じ位置に追加する。

                ELSE IF $w' == 1$

                ~~~~粒子数に変化なし。

                ELSE

                ~~~~その粒子を取り除く。

                (分岐過程に対応)

                END IF
            \end{itemize}

        \end{enumerate}

    \end{frame}

    \begin{frame}
        \begin{enumerate}
            \setcounter{enumi}{2}
            \item $V_{ave}$と$V_{ref}$の更新

            $V_{ave} = \frac{1}{N}\sum_{i} n_i V(x_i)$

            $V_{ref} = V_{ave} - \frac{a}{N_0\Delta\tau}(N-N_0)$

            \item 上記$2$と$3$を繰り返し、$V_{ave}$の平均を取ることで基底状態のエネルギーを求めることができる。
        \end{enumerate}
        同様に、$\hbar = m = 1$という単位系を採用し、
        $(\Delta s)^2 = 2D\Delta \tau~(\tau \ll 1)$

        また、$N$は上記$2$によって増減した後の粒子数であり、$a$は粒子数$N$が一定になるように調整するためのパラメーターである。
    \end{frame}

    \subsection{問題点}

    \begin{frame}{\insertsubsection}
      分岐過程において、(\ref{Gbranch})より
      \begin{equation}
        G_{branch/2}(x',x,\Delta\tau) = e^{-(\frac{1}{2}[V(x)+V(x')]-V_{ref})\Delta\tau} \nonumber
      \end{equation}
      この
      \begin{equation}
        V(x)+V(x')
      \end{equation}
      部分に問題がある。
      \vfill
      ポテンシャルが負に大きくなる場合に、粒子の複製数が非常に大きくなってしまう。
      (例:電子が原子核に近づいた時のクーロンポテンシャル)

      $\to$ 重み付き標本抽出を導入することで改善できる。
    \end{frame}


    \section{拡散量子モンテカルロ法2}

    \subsection{重み付き標本抽出の導入の準備}
    \begin{frame}{\insertsubsection}
      \begin{itemize}
        \item 試行関数$\Psi_T(x)$を導入することで、粒子が$V(x)$のより重要な方向に動くようにする。
      \end{itemize}

    \end{frame}

    \begin{frame}
      \begin{equation}
        f(x,\tau) = \Psi(x,\tau)\Psi_T(x)
      \end{equation}
      を導入する。

      これは、微分方程式
      \begin{equation}
        \frac{\partial f(x,\tau)}{\partial \tau} = D\frac{\partial^2 f}{\partial x^2} - D\frac{\partial [f(x,\tau)F(x)]}{\partial x} - (E_L(x) - V_{ref})f(x,\tau)
      \end{equation}
      を満たす。

      $E_L(x)$は局所エネルギー
      \begin{equation}
        E_L(x) = \frac{\hat{H}\Psi_T}{\Psi_T} = V(x) - \frac{D}{\Psi_T}\frac{\partial^2 \Psi_T}{\partial x^2}
      \end{equation}

      $F(x)$は
      \begin{equation}
        F(x) = \frac{2}{\Psi_T}\frac{\partial \Psi_T}{\partial x}
      \end{equation}
      この$F$によって粒子が$|\Psi_T|^2$の領域から離れていくこと(ドリフト)を意味している。
    \end{frame}

    \begin{frame}
      このドリフト項を組み込むと、
      \begin{equation}
        G_{diff}(x',x,\Delta\tau) = \dfrac{1}{\sqrt{4\pi D\Delta\tau}}
        e^{-\frac{(x'-x - d\tau F(x))^2}{4D\Delta\tau}}
      \end{equation}
      となるが、$x$と$x'$の対称性を崩してしまう。

      $\to$位置の更新がおかしくなってしまう。

      そのため、位置の更新にメトロポリス・アルゴリズムを組み込むことで修正する。

    \end{frame}

    \begin{frame}
      位置の更新を受け入れる確率を$p$とする。
      \begin{equation}
        p = \frac{|\Psi_T(x')|^2 G_{diff}(x,x',\Delta\tau)}{|\Psi_T(x)|^2G_{diff}(x',x,\Delta\tau)}
      \end{equation}

      $p \geq r$であれば位置を更新し、そうでなければ位置を更新しない。
      ($r$は$[0,1]$の一様乱数)

      また、分岐過程における$G_{branch}$は

      \begin{equation}
        G_{branch/2}(x',x,\Delta\tau) = e^{-(\frac{1}{2}[E_T(x)+E_T(x')]-V_{ref})\Delta\tau p}
      \end{equation}

      となる。

      ここで、$\Delta\tau p$を実効的な分割時間という。

      実効的な分割時間を用いるのは、拡散過程が起こらない場合があるためである。

      局所エネルギー$E_L$の平均値が基底状態のエネルギーになる。
    \end{frame}

    \subsection{アルゴリズム}
    \begin{frame}{\insertsubsection}
        \begin{enumerate}
            \item 初期化
            \begin{itemize}
                \item $N_0$個の粒子を初期値$x_i$にランダムに置く。
                \item $V_{ave} = V_{ref} = \frac{1}{N_0} \sum_{i=1}^{N_0}V(x_i)$ を計算。
            \end{itemize}

            \item 全ての粒子について
            \begin{itemize}
                \item 分散$2D\tau$、平均$0$のガウス分布から乱数$r$を生成する。
                \item $x_{new} = x_{old} + r + \Delta\tau F(x)$と新しい位置を決める。($G_{diff}$による拡散過程に対応)
                \item 更新確率$p$を計算し、メトロポリス・アルゴリズムに従って位置を更新する。
                \item 更新した位置で$w = G_{branch}$を計算する

            \end{itemize}

        \end{enumerate}

    \end{frame}

    \begin{frame}
        \begin{enumerate}
            \setcounter{enumi}{1}

            \item 単位区間$[0,1]$から乱数$r$を生成する。$w+r$の整数部分を$w'$とする。

            IF $w' > 1$

            ~~~~粒子を$w'-1$個同じ位置に追加する。

            ELSE IF $w' == 1$

            ~~~~粒子数に変化なし。

            ELSE

            ~~~~その粒子を取り除く。

            (分岐過程に対応)

            END IF

            \item $V_{ave}$と$V_{ref}$の更新

            $V_{ave} = \frac{1}{N}\sum_{i} n_i E_L(x_i)$

            $V_{ref} = V_{ave} - \frac{a}{N_0\Delta\tau}(N-N_0)$

            \item 上記$2$と$3$を繰り返し、$V_{ave}$の平均を取ることで基底状態のエネルギーを求めることができる。
        \end{enumerate}
        % 同様に、$\hbar = m = 1$という単位系を採用し、
        % $(\Delta s)^2 = 2D\Delta \tau~(\tau \ll 1)$
        %
        % また、$N$は上記$2$によって増減した後の粒子数であり、$a$は粒子数$N$が一定になるように調整するためのパラメーターである。
    \end{frame}

    \section{拡散量子モンテカルロ法3}

    \subsection{他の改善方法}
    \begin{frame}
      重み付き標本抽出の他にもアルゴリズムを改善する方法がある。

      \begin{itemize}
        \item 粒子数を変化させる分岐過程で、粒子の重みを変更することである。
      \end{itemize}
      具体的に$n$ステップ後の,$k$番目の粒子が重みが
      \begin{equation}
        W_n^k = \prod_{i = 1}^n G_{branch}^{(i,k)}
      \end{equation}
      となるようにする。

      $G_{branch}^{(i,k)}$は$i$ステップでの$k$番目の粒子の$G_{branch}$である。

    \end{frame}

    \subsection{アルゴリズム}
    \begin{frame}{\insertsubsection}
      アルゴリズムは単純で、拡散量子モンテカルロ$1$でのアルゴリズムにおいて、
      粒子の重み$w$を$W_n^k$に置き換えれば良い。

      ただし、粒子を複製する処理で、重み$W$を持った粒子が$n$個に複製される時、
      それらの粒子の重みは$W/n$になることに注意せよ。
    \end{frame}





\end{document}
