\documentclass[dvipdfmx]{beamer}

%Beamerの設定
% \usetheme{Boadilla}
\usetheme{Antibes}

%Beamerフォント設定
\usepackage{newtxtext,newtxmath} % TXフォント
\usepackage{sfmath}
\DeclareSymbolFont{operators}{OT1}{\sfdefault}{m}{n}
\SetSymbolFont{operators}{bold}{OT1}{\sfdefault}{b}{n}
\usefonttheme{structurebold} % タイトル部を太字
\setbeamerfont{alerted text}{series=\bfseries} % Alertを太字
\setbeamerfont{section in toc}{series=\mdseries} % 目次は太字にしない
\setbeamerfont{frametitle}{size=\Large} % フレームタイトル文字サイズ
\setbeamerfont{title}{size=\LARGE} % タイトル文字サイズ
\setbeamerfont{date}{size=\small}  % 日付文字サイズ

% Babel (日本語の場合のみ・英語の場合は不要)
\uselanguage{japanese}
\languagepath{japanese}
\deftranslation[to=japanese]{Theorem}{定理}
\deftranslation[to=japanese]{Lemma}{補題}
\deftranslation[to=japanese]{Example}{例}
\deftranslation[to=japanese]{Examples}{例}
\deftranslation[to=japanese]{Definition}{定義}
\deftranslation[to=japanese]{Definitions}{定義}
\deftranslation[to=japanese]{Problem}{問題}
\deftranslation[to=japanese]{Solution}{解}
\deftranslation[to=japanese]{Fact}{事実}
\deftranslation[to=japanese]{Proof}{証明}
\def\proofname{証明}

%Beamer色設定
\definecolor{UniBlue}{RGB}{0,150,200}
\definecolor{AlertOrange}{RGB}{255,76,0}
\definecolor{AlmostBlack}{RGB}{38,38,38}
\setbeamercolor{normal text}{fg=AlmostBlack}  % 本文カラー
\setbeamercolor{structure}{fg=UniBlue} % 見出しカラー
% \setbeamercolor{block title}{fg=UniBlue!50!black} % ブロック部分タイトルカラー
\setbeamercolor{alerted text}{fg=AlertOrange} % \alert 文字カラー
\mode<beamer>{
    \definecolor{BackGroundGray}{RGB}{254,254,254}
    \setbeamercolor{background canvas}{bg=BackGroundGray} % スライドモードのみ背景をわずかにグレーにする
}

%フラットデザイン化
\setbeamertemplate{blocks}[rounded] % Blockの影を消す
\useinnertheme{circles} % 箇条書きをシンプルに
\setbeamertemplate{navigation symbols}{} % ナビゲーションシンボルを消す
\setbeamertemplate{footline}[frame number] % フッターはスライド番号のみ

% section毎に目次の表示
\AtBeginSection[]{
    \begin{frame}
        \tableofcontents[currentsection]
    \end{frame}
}

\title{量子力学とモンテカルロ法}
\author{齊藤}

\begin{document}

\maketitle
\frame{\tableofcontents[hideallsubsections]}

\section{目的}
\begin{frame}{目的}
    シュレディンガー方程式を解いて、調和振動子ポテンシャルや水素原子などの基底状態・励起状態のエネルギーを求めたい。

\end{frame}


\section{量子力学 復習}
\subsection{シュレディンガー方程式}
\begin{frame}{\insertsubsection}
    ポテンシャルが時間に依存する場合
    \begin{block}{時間に依存するシュレディンガー方程式}
        \begin{equation}
            i\hbar\dfrac{\partial \Psi(x,t)}{\partial t} = -\dfrac{\hbar^2}{2m}\dfrac{\partial^2 \Psi(x,t)}{\partial x^2} + V(x,t)\Psi(x,t)
        \end{equation}
    \end{block}

    時間に関して$1$階、位置に関して$2$階の偏微分方程式
    \end{frame}

    \begin{frame}
    ポテンシャルが時間に依存しない場合
        \begin{equation}
            i\hbar\dfrac{\partial \Psi(x,t)}{\partial t} = -\dfrac{\hbar^2}{2m}\dfrac{\partial^2 \Psi(x,t)}{\partial x^2} + V(x)\Psi(x,t)
        \end{equation}
    波動関数を時間と位置の関数の積として考えると

        \begin{equation}
            \Psi(x,t) = \psi(x)\phi(t)
        \end{equation}
    変数分離ができて、時間に関しては
        \begin{equation}
            \phi(t) = e^{-i\frac{E}{\hbar}t}
        \end{equation}
    位置に関しては
    \begin{block}{時間に依存しないシュレディンガー方程式}
        \begin{equation}
            \left[-\dfrac{\hbar^2}{2m}\dfrac{d^2}{dx^2} + V(x)\right]\psi(x) = E\psi(x)
        \end{equation}
    \end{block}
    \end{frame}

    \begin{frame}
        \begin{block}{ハミルトニアン$\hat{H}$}
            \begin{equation}
                \hat{H} = -\dfrac{\hbar^2}{2m}\dfrac{d^2}{dx^2} + V(x)
            \end{equation}
        \end{block}

        とすると、時間に依存しないシュレディンガー方程式は

        \begin{block}{時間に依存しないシュレディンガー方程式}
            \begin{equation}
                \hat{H}\psi(x) = E\psi(x)
            \end{equation}
        \end{block}
        つまり、
        \alert{$\psi$はハミルトニアン$\hat{H}$の固有値$E$に対応する固有関数}

    \end{frame}

    \begin{frame}
        (時間に依存しない)シュレディンガー方程式を解くとは、

        \alert{
        ポテンシャル$V$によって変化するハミルトニアン$\hat{H}$の固有値$E$と対応する固有関数$\psi$を求める
        }

        ということ。(固有値問題)
    \end{frame}

    \subsection{物理量の観測}
    \begin{frame}{\insertsubsection}

        古典力学の世界では、ある時刻での粒子の位置と速度を与えると、全ての時刻での位置と運動量を完全に決定できる。
        つまり、現在の粒子の情報がわかれば未来での粒子の情報を予言できる。$\mathcal{O}(h)$が無視できる。
        \vfill
        しかし、量子力学の世界では$\mathcal{O}(h)$が無視できないため、不確定性原理の影響が出てくる。
        つまり、位置と運動量の両方を同時に決めることができない。
        \vfill
        粒子がどこに存在するのか、その分布を表すものが波動関数$\psi$の絶対値の$2$乗$|\psi|^2$である。
        これを確率密度関数と呼び、正規化されていれば
        \begin{equation}
            \int_{-\infty}^\infty |\psi|^2 dx = 1
        \end{equation}

        % 波動関数$\psi$は直交関数系を成し、その絶対値の$2$乗$|\psi|^2$は$1$つの粒子の存在の確率密度を表す。
        % $\psi$が規格化されていれば、
        % \begin{equation}
        %     \int_{-\infty}^\infty |\psi|^2 dx = 1
        % \end{equation}

    \end{frame}

    \begin{frame}
        観測したい物理量(位置、運動量、エネルギー)は実際に観測できるのだから実数であるはずだ。
        物理量$A$に対応する演算子を$\hat{A}$とすると、物理量$A$の期待値$\langle A \rangle$は
        \begin{equation}
            \langle A \rangle = \int \psi^*(x) \hat{A} \psi(x) dx
        \end{equation}
        と求めることができる。
    \end{frame}

    \begin{frame}
        ここで、ハミルトニアン$\hat{H}$の期待値$\langle H \rangle$を求める。

        \begin{equation}
            \langle H \rangle = \int_{-\infty}^\infty \psi^*(x) \hat{H} \psi(x) dx
        \end{equation}
        時間に依存しないシュレディンガー方程式$\hat{H}\psi = E\psi$より
        \begin{equation}
            \langle H \rangle = \int_{-\infty}^\infty \psi^*(x) E \psi(x) dx
        \end{equation}
        $E$は定数なので、
        \begin{equation}
            \langle H \rangle = E\int_{-\infty}^\infty \psi^*(x)\psi(x) dx = E
        \end{equation}
        となる。
        つまり、
        \alert{ハミルトニアン$\hat{H}$の期待値$\langle H \rangle$はエネルギー$E$}である。
    \end{frame}

    \section{モンテカルロ法}
    \begin{frame}
        \begin{itemize}
            \item 数値計算を(擬似)乱数を用いて行う手法の総称。
            \item 今回は数値積分
            % \item 高次元の数値積分に強いが次元の呪いがある。
        \end{itemize}
    \end{frame}

    \subsection{当たり外れ法}
    \begin{frame}{\insertsubsection}
        面積が不等式で評価されるような場合に有効
        \begin{example}
            \begin{equation}
                x^2 + y^2 -1 < 0
            \end{equation}
            で評価される面積
        \end{example}

        \begin{figure}
            \includegraphics[width=70mm]{../fig/monte1.png}
            % \caption{}
            % \label{}
        \end{figure}
    \end{frame}

    \subsection{標本平均法}
    \begin{frame}{\insertsubsection}
        \begin{Theorem}[積分の平均値の定理]
            関数$f(x)$が区間$[a,b]$で連続かつ微分可能であれば
            \begin{equation}
                \dfrac{1}{b-a}\int_a^b f(x) dx = f(k)
            \end{equation}
            となるような$k,(a\leq k \leq b)$が存在する。
        \end{Theorem}
        これより、区間$[a,b]$の乱数列を$x_i$とすると
        \begin{equation}
            \dfrac{1}{b-a}\int_a^b f(x) dx = \lim_{n\to\infty}\dfrac{1}{n}\sum_{i=0}^n f(x_i)
        \end{equation}
        このように積分できる。
    \end{frame}

    \begin{frame}
        \begin{example}
            \begin{equation}
                4\int_0^1 \sqrt{1-x^2} dx
            \end{equation}
        \end{example}

        \begin{figure}
            \includegraphics[width=70mm]{../fig/monte2.png}
            % \caption{}
            % \label{}
        \end{figure}
    \end{frame}

    \subsection{一般的な話}
    \begin{frame}{\insertsubsection}

        \begin{itemize}
            \item 標本空間を$\Omega \subset \mathbb{R}^d$
            \item 確率密度関数を$p(x)$
        \end{itemize}

        として、関数$f(x)$の期待値$E[f(x)]$を考える。

        これは、
        \begin{equation}
            E[f(x)] = \int_\Omega f(x)p(x) dx
        \end{equation}
        で計算できる。

    \end{frame}

    \begin{frame}
        確率密度関数$p(x)$によって定義される確率分布に従って生成される乱数列を$x_i$とすると期待値$E[f(x)]$は
        \begin{equation}
            E[f(x)] = \lim_{n \to \infty}\dfrac{1}{n}\sum_{i=1}^n f(x_i)
        \end{equation}

        と計算できる。
        つまり、
        \begin{equation}
            E[f(x)] = \int_\Omega f(x)p(x) dx = \lim_{n \to \infty}\dfrac{1}{n}\sum_{i=1}^n f(x_i)
        \end{equation}

    \end{frame}

    \subsection{具体例}
    \begin{frame}{\insertsubsection}
        $1$次元で具体例を考えてみる。

        \begin{itemize}
            \item 標本空間を$[a,b]$
            \item 確率変数を$x \in [a,b]$
            \item $[a,b]$で一様分布の確率密度関数は$p(x) = \frac{1}{b-a}$
        \end{itemize}

        関数$f(x)$の期待値$E[f(x)]$は

        \begin{equation}
            E[f(x)] = \dfrac{1}{b-a}\int_a^b f(x) dx = \lim_{n\to\infty}\dfrac{1}{n}\sum_{i=0}^n f(x_i)
        \end{equation}

        \alert{標本平均法に一致する!}

    \end{frame}

    \subsection{誤差解析}
    \begin{frame}{\insertsubsection}

        \begin{itemize}
            \item $f(x_1),f(x_2),\cdots,f(x_N)$を平均$\mu$、分散$\sigma^2$の母集団からの無作為標本
            \item 標本平均$\bar{f} = \dfrac{1}{N} \sum_{i=1}^{N}f(x_i)$
            \item 標本平均の分散$V[\bar{f}] = \dfrac{1}{N}\sum_{i=1}^{N} V[\bar{f_i}] = \dfrac{\sigma^2}{N}$
            \item 標本偏差 $ = \dfrac{\sigma}{\sqrt{N}}$
        \end{itemize}

        つまり、
        \alert{母集団の偏差$\sigma$がどうであれ、$N$を十分大きくすれば標本偏差を小さくすることができる。}
        また、\alert{$\sigma$を小さくできれば同じ$N$でも誤差が小さい。}

    \end{frame}

    \begin{frame}
        標本偏差$\sigma$を小さくするために、
        \begin{itemize}
            \item 重み付き標本抽出
            \item マルコフ連鎖
                \begin{itemize}
                    \item メトロポリス・ヘイスティング法
                    \item ギブスサンプリング
                \end{itemize}
        \end{itemize}
        などがある。
    \end{frame}

    \section{変分法}
    \begin{frame}
        ハミルトニアン$\hat{H}$の期待値$\langle H \rangle$
        \begin{equation}
            \langle H \rangle = \int_{-\infty}^\infty \psi^*(x) \hat{H} \psi(x) dx
        \end{equation}
        を計算したい。

        しかし、$\psi(x)$は未知の関数。
    \end{frame}

    \begin{frame}
        試行関数$\psi_{trial}(x,a)$というものを用意する。

        パラメーター$a$によって関数が変化する。

        \begin{equation}
            \int_{-\infty}^\infty \psi_{trial}^*(x,a) \hat{H} \psi_{trial}(x,a) dx = E(a)
        \end{equation}

        エネルギー$E$がパラメーター$a$によって変化する。

        $E(a)$の極値を$E'$、基底状態のエネルギーを$E_0$とすると、

        \begin{equation}
            E' \geq E_0
        \end{equation}

        が成立する。(変分原理)

        このように、\alert{$E(a)$の極値を求めることで、基底状態のエネルギーを近似的に求めることができる。}

        ただし、試行関数$\psi_{trial}$をいい感じに選ぶ必要がある。

    \end{frame}

    \section{変分モンテカルロ法}
    \begin{frame}
        試行関数を$\Psi_{tiral}(x,a)$、基底状態のエネルギーを$E_0$とすると、変分原理より
        \begin{equation}
            \langle H \rangle = E[\Psi] = \dfrac{\int\Psi^* \hat{H} \Psi dx}{\int \Psi^* \Psi dx} \geq E_0
        \end{equation}

        パラメーター$a$を変化させて数値積分をして得られる$E[\Psi]$が最小となるものを見つける。
        \vfill
        この数値積分が応用上多次元になるのでモンテカルロ法を用いる。
        \vfill
        束縛状態では$\Psi^* = \Psi$と考えて良い。
    \end{frame}

    \begin{frame}
        重み付き標本抽出を使える形にする。
        \begin{equation}
            E[\Psi] = \dfrac{\int \Psi^2(x) E_L(x) dx}{\int \Psi^2(x) dx}
        \end{equation}
        ここで、$E_L$は局所エネルギー
        \begin{equation}
            E_L = \dfrac{\hat{H}\Psi(x)}{\Psi(x)}
        \end{equation}
        これを用いることで、
        \begin{equation}
            E[\Psi] = \lim_{n\to\infty}\dfrac{1}{n}\sum_{i=1}^n E_L(x_i)
        \end{equation}

        $x_i$は確率密度関数$\Psi^2$によって定義される分布によって従う乱数列である。
    \end{frame}

    \section{シュレディンガー方程式と拡散方程式}

    \begin{frame}
        シュレディンガー方程式
        \begin{equation}
            i\hbar\dfrac{\partial \Psi(x,t)}{\partial t} = -\dfrac{\hbar^2}{2m}\dfrac{\partial^2 \Psi(x,t)}{\partial x^2} + V(x)\Psi(x,t)
        \end{equation}

        $\tau = \dfrac{it}{\hbar}$と変数変換すると、

        \begin{block}{虚時間のシュレディンガー方程式}
            \begin{equation}
                \label{tmp1}
        \dfrac{\partial \Psi(x,\tau)}{\partial \tau} = \dfrac{\hbar^2}{2m}\dfrac{\partial^2 \Psi(x,\tau)}{\partial x^2} - V(x)\Psi(x,\tau)
            \end{equation}
        \end{block}
        これと、
        \begin{block}{拡散方程式}
            \begin{equation}
                \dfrac{\partial f(x,t)}{\partial t} = D \dfrac{\partial^2 f(x,t)}{\partial x^2}
            \end{equation}
        \end{block}
        似ていることがわかる。
    \end{frame}

    \begin{frame}

        \begin{block}{虚時間のシュレディンガー方程式}
            \begin{equation}
                \label{tmp1}
        \dfrac{\partial \Psi(x,\tau)}{\partial \tau} = \dfrac{\hbar^2}{2m}\dfrac{\partial^2 \Psi(x,\tau)}{\partial x^2} - V(x)\Psi(x,\tau)
            \end{equation}
        \end{block}

        \begin{itemize}
            \item 右辺第$1$項は粒子の拡散 (拡散過程)
            \item 右辺第$2$項はポテンシャル$V$の符号によって粒子が増減(分岐過程)
        \end{itemize}

        を表している。

        つまり、
        \alert{シュレディンガー方程式を虚時間に発展する拡散方程式と見て解くことができる。}
    \end{frame}


    \section{ランダムウォーク量子モンテカルロ法}
    \begin{frame}{準備}
        拡散方程式を解く方法の$1$つにランダムウォークを用いる方法がある。
        粒子を空間上に配置し、粒子をランダムに移動(拡散)させることで、系の状態を定常状態にしていく。つまり、基底状態のエネルギー$E_0$を求めることができる。
        \vfill
        しかし、(\ref{tmp1})の右辺第$2$項の影響で粒子が激しく増加または減少してしまう。粒子数を一定にするために、$V_{ref}$という参照ポテンシャルを用意し、これを用いて系のエネルギーを求める。
    \end{frame}

    \begin{frame}{アルゴリズム}
        \begin{enumerate}
            \item 初期化
            \begin{itemize}
                \item $N_0$個の粒子を初期値$x_i$にランダムに置く。
                \item $V_{ave} = V_{ref} = \frac{1}{N_0} \sum_{i=1}^{N_0}V(x_i)$ を計算。
            \end{itemize}
            \item 全ての粒子について
            \begin{itemize}
                \item 参照エネルギー
                $V_{ref} = \frac{1}{N_0}\sum_{i=1}^{N_0}V(x[i])$
                を計算。
                \item 粒子をランダムに$\Delta s$だけ移動(拡散)させる。

                $x[i] = x[i] \pm \Delta s$

                \item $\Delta V = V(x_i) - V_{ref}$を計算。
                $r$を単位区間$[0,1]$の乱数とする。

                IF $\Delta V > 0$ and $r < \Delta V \Delta \tau$

                ~~~~ その粒子を取り除く。

                ELSE IF $\Delta V < 0$ and $r < -\Delta V \Delta \tau$

                ~~~~ 粒子を$1$個同じ位置に追加する。

                END IF
            \end{itemize}

        \end{enumerate}
    \end{frame}

    \begin{frame}
        \begin{enumerate}
            \setcounter{enumi}{2}
            \item $V_{ave}$と$V_{ref}$の更新

            $V_{ave} = \frac{1}{N}\sum_{i=1}^{N}V(x_i)$

            $V_{ref} = V_{ave} - \frac{a}{N_0\Delta\tau}(N-N_0)$

            \item 上記$2$と$3$が$V_{ave}$がある値で不規則なゆらぎを示すまで繰り返す。

            \item $V_{ave}$の平均を取ることで基底状態のエネルギーを求めることができる。
        \end{enumerate}
        ここで、$\hbar = m = 1$という単位系を採用し、
        $(\Delta s)^2 = 2D\Delta \tau~(\tau \ll 1)$

        また、$N$は上記$2$によって増減した後の粒子数であり、$a$は粒子数$N$が一定になるように調整するためのパラメーターである。
    \end{frame}

    \section{拡散量子モンテカルロ法}
    \begin{frame}{準備}
        \begin{block}{グリーン関数$G(x',x,\tau)$}
            \begin{equation}
                \Psi(x',\tau) = \int G(x',x,\tau)\Psi(x,0)dx
            \end{equation}
        \end{block}

        $\Psi$を時刻$0$から時刻$\tau$へ伝播させる。

        \begin{equation}
            \label{tmp2}
            \dfrac{\partial G}{\partial \tau} = -(\hat{H} - V_{ref})G
        \end{equation}
        や
        \begin{equation}
            G(x',x,\tau) = G(x,x',\tau)
        \end{equation}
        を満たす。
    \end{frame}

    \begin{frame}
        (\ref{tmp2})の形式解は
        \begin{equation}
            G(\tau) = e^{-(\hat{H} - V_{ref})\tau}
        \end{equation}
        演算子の指数関数は、そのテイラー展開を意味する。

        というのも、量子力学において演算子は無限次元のベクトル(波動関数)に作用する無限次元の行列と考えられるため。
        (ハイゼンベルグが考えた行列力学というものがある。)

        \begin{block}{行列の指数関数}
            \begin{equation}
                e^A = \sum_{k=0}^\infty \dfrac{1}{k!}A^k
            \end{equation}
        \end{block}
    \end{frame}

    \begin{frame}
        $\hat{H}$の中にある運動エネルギー演算子$\hat{T}(=\frac{\hat{p}^2}{2m},\hat{p} = -i\hbar\frac{\partial}{\partial x})$
        とポテンシャルエネルギー演算子$\hat{V}$が互いに非可換であるため、演算子の指数関数の計算が困難。

        そのため
        \begin{block}{鈴木・トロッターの分解公式}
            \begin{align}
                G_{diff} &= e^{-\hat{T}\Delta\tau} \\
                G_{branch/2} &= e^{-\frac{1}{2}(\hat{T} - V_{ref})\Delta\tau} \\
                G(\tau) &= G_{branch/2}G_{diff}G_{branch/2}
            \end{align}
        \end{block}
        を用いる。

        $G_{diff}$が拡散過程、$G_{branch/2}$が分岐過程を表す。

        つまり、$G$は分岐$\to$拡散$\to$分岐の流れを表している。
    \end{frame}

    \begin{frame}
        $G_{diff}$は
        \begin{equation}
            \dfrac{\partial G_{diff}}{\partial \tau} = - \hat{T}G_{diff} = \dfrac{\hbar^2}{2m}\dfrac{\partial^2 G_{diff}}{\partial x^2}
        \end{equation}
        $G_{branch/2}$は
        \begin{equation}
            \dfrac{\partial G_{branch/2}}{\partial \tau} = -\dfrac{1}{2}(\hat{V} - V_{ref})G_{branch/2}
        \end{equation}
        の微分方程式を満たす。

        これらの解は

        \begin{align}
            G_{diff}(x',x,\Delta\tau) &= \dfrac{1}{\sqrt{4\pi D\Delta\tau}}
            e^{-\frac{(x'-x)^2}{4D\Delta\tau}} \\
            G_{branch/2}(x',x,\Delta\tau) &= e^{-(\frac{1}{2}[V(x)+V(x')]-V_{ref})\Delta\tau}
        \end{align}
        ただし、$D=\dfrac{\hbar^2}{2m}$である。
    \end{frame}

    \begin{frame}{アルゴリズム}
        \begin{enumerate}
            \item 初期化
            \begin{itemize}
                \item $N_0$個の粒子を初期値$x_i$にランダムに置く。
                \item $V_{ave} = V_{ref} = \frac{1}{N_0} \sum_{i=1}^{N_0}V(x_i)$ を計算。
            \end{itemize}

            \item 全ての粒子について
            \begin{itemize}
                \item 分散$2D\tau$、平均$0$のガウス分布から乱数$r$を生成する。
                \item $x_i = x_i + r$と位置を更新する。($G_{diff}$による拡散過程に対応)
                \item $w(x\to x',\tau) = e^{-(\frac{1}{2}[V(x)+V(x')]-V_{ref})\Delta\tau}$を計算する。
                \item 単位区間$[0,1]$から乱数$r$を生成する。$w+r$の整数部分を$w'$とする。

                IF $w' > 1$

                ~~~~粒子を$w'-1$個同じ位置に追加する。

                ELSE IF $w' == 1$

                ~~~~粒子数に変化なし。

                ELSE

                ~~~~その粒子を取り除く。

                (分岐過程に対応)

                END IF
            \end{itemize}

        \end{enumerate}

    \end{frame}

    \begin{frame}
        \begin{enumerate}
            \setcounter{enumi}{2}
            \item $V_{ave}$と$V_{ref}$の更新

            $V_{ave} = \frac{1}{N}\sum_{i=1}^{N}V(x_i)$

            $V_{ref} = V_{ave} - \frac{a}{N_0\Delta\tau}(N-N_0)$

            \item 上記$2$と$3$を繰り返し、$V_{ave}$の平均を取ることで基底状態のエネルギーを求めることができる。
        \end{enumerate}
        同様に、$\hbar = m = 1$という単位系を採用し、
        $(\Delta s)^2 = 2D\Delta \tau~(\tau \ll 1)$

        また、$N$は上記$2$によって増減した後の粒子数であり、$a$は粒子数$N$が一定になるように調整するためのパラメーターである。
    \end{frame}

\end{document}
