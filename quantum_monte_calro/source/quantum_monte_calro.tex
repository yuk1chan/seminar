\documentclass[dvipdfmx]{beamer}

%Beamerの設定
\usetheme{Boadilla}

%Beamerフォント設定
\usepackage{newtxtext,newtxmath} % TXフォント
\usepackage{sfmath}
\DeclareSymbolFont{operators}{OT1}{\sfdefault}{m}{n}
\SetSymbolFont{operators}{bold}{OT1}{\sfdefault}{b}{n}
\usefonttheme{structurebold} % タイトル部を太字
\setbeamerfont{alerted text}{series=\bfseries} % Alertを太字
\setbeamerfont{section in toc}{series=\mdseries} % 目次は太字にしない
\setbeamerfont{frametitle}{size=\Large} % フレームタイトル文字サイズ
\setbeamerfont{title}{size=\LARGE} % タイトル文字サイズ
\setbeamerfont{date}{size=\small}  % 日付文字サイズ

% Babel (日本語の場合のみ・英語の場合は不要)
\uselanguage{japanese}
\languagepath{japanese}
\deftranslation[to=japanese]{Theorem}{定理}
\deftranslation[to=japanese]{Lemma}{補題}
\deftranslation[to=japanese]{Example}{例}
\deftranslation[to=japanese]{Examples}{例}
\deftranslation[to=japanese]{Definition}{定義}
\deftranslation[to=japanese]{Definitions}{定義}
\deftranslation[to=japanese]{Problem}{問題}
\deftranslation[to=japanese]{Solution}{解}
\deftranslation[to=japanese]{Fact}{事実}
\deftranslation[to=japanese]{Proof}{証明}
\def\proofname{証明}

%Beamer色設定
\definecolor{UniBlue}{RGB}{0,150,200}
\definecolor{AlertOrange}{RGB}{255,76,0}
\definecolor{AlmostBlack}{RGB}{38,38,38}
\setbeamercolor{normal text}{fg=AlmostBlack}  % 本文カラー
\setbeamercolor{structure}{fg=UniBlue} % 見出しカラー
\setbeamercolor{block title}{fg=UniBlue!50!black} % ブロック部分タイトルカラー
\setbeamercolor{alerted text}{fg=AlertOrange} % \alert 文字カラー
\mode<beamer>{
    \definecolor{BackGroundGray}{RGB}{254,254,254}
    \setbeamercolor{background canvas}{bg=BackGroundGray} % スライドモードのみ背景をわずかにグレーにする
}

%フラットデザイン化
\setbeamertemplate{blocks}[rounded] % Blockの影を消す
\useinnertheme{circles} % 箇条書きをシンプルに
\setbeamertemplate{navigation symbols}{} % ナビゲーションシンボルを消す
\setbeamertemplate{footline}[frame number] % フッターはスライド番号のみ

% section毎に目次の表示
\AtBeginSection[]{
    \begin{frame}
        \tableofcontents[currentsection]
    \end{frame}
}

\title{量子モンテカルロ法}
\author{齊藤}

\begin{document}

\maketitle
\frame{\tableofcontents[hideallsubsections]}

\section{量子力学 復習}
\subsection{シュレディンガー方程式}

\begin{frame}{\insertsubsection}
    ポテンシャルが時間に依存する場合
    \begin{block}{時間に依存するシュレディンガー方程式}
        \begin{equation}
            i\hbar\dfrac{\partial \Psi(x,t)}{\partial t} = -\dfrac{\hbar^2}{2m}\dfrac{\partial^2 \Psi(x,t)}{\partial x^2} + V(x,t)\Psi(x,t)
        \end{equation}
    \end{block}

    時間に関して$1$階、位置に関して$2$階の偏微分方程式
    \end{frame}

    \begin{frame}{\insertsubsection}
    ポテンシャルが時間に依存しない場合
        \begin{equation}
            i\hbar\dfrac{\partial \Psi(x,t)}{\partial t} = -\dfrac{\hbar^2}{2m}\dfrac{\partial^2 \Psi(x,t)}{\partial x^2} + V(x)\Psi(x,t)
        \end{equation}
    波動関数を時間と位置の関数の積として考えると

        \begin{equation}
            \Psi(x,t) = \psi(x)\phi(t)
        \end{equation}
    変数分離ができて、時間に関しては
        \begin{equation}
            \phi(t) = e^{-i\frac{E}{\hbar}t}
        \end{equation}
    位置に関しては
    \begin{block}{時間に依存しないシュレディンガー方程式}
        \begin{equation}
            \left[-\dfrac{\hbar^2}{2m}\dfrac{d^2}{dx^2} + V(x)\right]\psi(x) = E\psi(x)
        \end{equation}
    \end{block}
    \end{frame}

    \begin{frame}{\insertsubsection}
        \begin{block}{ハミルトニアン$\hat{H}$}
            \begin{equation}
                \hat{H} = -\dfrac{\hbar^2}{2m}\dfrac{d^2}{dx^2} + V(x)
            \end{equation}
        \end{block}

        とすると、時間に依存しないシュレディンガー方程式は

        \begin{block}{時間に依存しないシュレディンガー方程式}
            \begin{equation}
                \hat{H}\psi(x) = E\psi(x)
            \end{equation}
        \end{block}
        つまり、
        \alert{$\psi$はハミルトニアン$\hat{H}$の固有値$E$に対応する固有関数}

    \end{frame}

    \begin{frame}{\insertsubsection}
        (時間に依存しない)シュレディンガー方程式を解くとは、

        \alert{
        ポテンシャル$V$によって変化するハミルトニアン$\hat{H}$の固有値$E$と対応する固有関数$\psi$を求める
        }

        ということ。(固有値問題)
    \end{frame}

    \subsection{物理量の観測}
    \begin{frame}{\insertsubsection}

        古典力学の世界では、ある時刻での粒子の位置と速度を与えると、全ての時刻での位置と運動量を完全に決定できる。
        つまり、現在の粒子の情報がわかれば未来での粒子の情報を予言できる。$\mathcal{O}(h)$が無視できる。
        \vfill
        しかし、量子力学の世界では$\mathcal{O}(h)$が無視できないため、不確定性原理の影響が出てくる。
        つまり、位置と運動量の両方を同時に決めることができない。
        \vfill
        粒子がどこに存在するのか、その分布を表すものが波動関数$\psi$の絶対値の$2$乗$|\psi|^2$である。
        これを確率密度関数と呼び、正規化されていれば
        \begin{equation}
            \int_{-\infty}^\infty |\psi|^2 dx = 1
        \end{equation}

        % 波動関数$\psi$は直交関数系を成し、その絶対値の$2$乗$|\psi|^2$は$1$つの粒子の存在の確率密度を表す。
        % $\psi$が規格化されていれば、
        % \begin{equation}
        %     \int_{-\infty}^\infty |\psi|^2 dx = 1
        % \end{equation}

    \end{frame}

    \begin{frame}{\insertsubsection}
        観測したい物理量(位置、運動量、エネルギー)は実際に観測できるのだから実数であるはずだ。
        物理量$A$に対応する演算子を$\hat{A}$とすると、物理量$A$の期待値$\langle A \rangle$は
        \begin{equation}
            \langle A \rangle = \int \psi^*(x) \hat{A} \psi(x) dx
        \end{equation}
        と求めることができる。
    \end{frame}

    \begin{frame}{\insertsubsection}
        ここで、ハミルトニアン$\hat{H}$の期待値$\langle H \rangle$を求める。

        \begin{equation}
            \langle H \rangle = \int_{-\infty}^\infty \psi^*(x) \hat{H} \psi(x) dx
        \end{equation}
        時間に依存しないシュレディンガー方程式$\hat{H}\psi = E\psi$より
        \begin{equation}
            \langle H \rangle = \int_{-\infty}^\infty \psi^*(x) E \psi(x) dx
        \end{equation}
        $E$は定数なので、
        \begin{equation}
            \langle H \rangle = E\int_{-\infty}^\infty \psi^*(x)\psi(x) dx = E
        \end{equation}
        となる。
        つまり、
        \alert{ハミルトニアン$\hat{H}$の期待値$\langle H \rangle$はエネルギー$E$}である。
    \end{frame}

    % \end{frame}{\insertsubsection}
    %
    % \begin{frame}

\end{document}
